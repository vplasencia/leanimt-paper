% \documentclass{} is the first command in any LaTeX code.  
% It is used to define what kind of document you are creating 
% such as an article or a book, and begins the document preamble
\documentclass{article} 

% Packages for customizing the ToC
\usepackage{tocloft}
\usepackage{hyperref}

% Add dots to the table of contents
\renewcommand{\cftsecleader}{\cftdotfill{\cftdotsep}}
\renewcommand{\cftsubsecleader}{\cftdotfill{\cftdotsep}}
\renewcommand{\cftsubsubsecleader}{\cftdotfill{\cftdotsep}}

% Adjust spacing in the ToC (optional)
\setlength{\cftbeforesecskip}{1.5pt}
\setlength{\cftbeforesubsecskip}{1.0pt}

% Hyperlink settings
\hypersetup{
    colorlinks=true,
    linkcolor=blue,
    filecolor=magenta,
    urlcolor=cyan,
    pdftitle={Custom Table of Contents},
    pdfpagemode=FullScreen,
}

% Article title
\title{LeanIMT: An optimized IMT} 
% Authors name
\author{Privacy \& Scaling Explorations} 
% Date for date compiled
\date{\today} 

% The preamble ends with the command \begin{document}
% All begin commands must be paired with an end command somewhere
\begin{document}
% Creates title using information in preamble (title, author, date)
\maketitle

% Creates a section for the Abstract
\section{Abstract}

% Insert a new page
\newpage
% Insert the Table of Contents
\tableofcontents
% Insert a new page
\newpage

% Creates a section for the Introduction
\section{Introduction}

% Creates a subsection for the Motivation within Introduction
\subsection{Motivation}

\section{LeanIMT Technical Explanation}

\section{Benchmarks}

\section{Conslusions}

% This is the end of the document
\end{document}