% \documentclass{} is the first command in any LaTeX code.  
% It is used to define what kind of document you are creating 
% such as an article or a book, and begins the document preamble
\documentclass{article} 

% Packages for customizing the ToC
\usepackage{tocloft}
\usepackage{xcolor} % Package for color definitions
\usepackage{hyperref}

% Create trees
\usepackage{forest}

% Create two column content
\usepackage{multicol}

% Add dots to the table of contents
\renewcommand{\cftsecleader}{\cftdotfill{\cftdotsep}}
\renewcommand{\cftsubsecleader}{\cftdotfill{\cftdotsep}}
\renewcommand{\cftsubsubsecleader}{\cftdotfill{\cftdotsep}}

% Define custom colors using xcolor
\definecolor{blue600}{HTML}{2563eb}
\definecolor{blue700}{HTML}{1d4ed8}

% Hyperlink settings
\hypersetup{
    % Enable colored links
    colorlinks=true,
    % Color for internal links (sections, pages, etc.)
    linkcolor=blue700,
    % Color for external URLs
    urlcolor=blue600,
    % Color for citation links
    citecolor=blue700
}

% Bibliography package
\usepackage[backend=biber,style=numeric]{biblatex}
\addbibresource{references.bib}

% Article title
\title{LeanIMT: An optimized IMT} 
% Authors name
\author{Privacy \& Scaling Explorations} 
% Date for date compiled
\date{\today} 

% The preamble ends with the command \begin{document}
% All begin commands must be paired with an end command somewhere
\begin{document}
% Creates title using information in preamble (title, author, date)
\maketitle

% Creates a section for the Abstract
\section{Abstract}

% Insert a new page
\newpage
% Insert the Table of Contents
\tableofcontents
% Insert a new page
\newpage

% Creates a section for the Introduction
\section{Introduction}

% Creates a subsection for the Motivation within Introduction
\subsection{Motivation}

\section{Merkle Tree}

\subsection{Incremental Merkle Tree}

An Incrememental Merkle Tree (IMT) is a Merkle Tree (MT) designed to be updated efficiently.

% align to the left
\raggedright

\bigbreak

\subsection{Binary Tree}

A Binary Tree is a tree data structure in which each node has at most two children, referred to as the left child and the right child.

TODO: Explain what is a Merkle tree and an Incremental Merkle Tree.

% Add a new empty line
\bigbreak

% Generic Merkle Tree 

\begin{center}
    \begin{forest}
        for tree={edge path={\noexpand\path[\forestoption{edge}] (\forestOve{\forestove{@parent}}{name}.parent anchor) -- +(0,-18pt)-| (\forestove{name}.child anchor)\forestoption{edge label};}, rectangle, draw, align=center}
        [$H_6$ \\ \color{blue600}H($H_4{||}H_5$)
        [$H_4$ \\ \color{blue600}H($H_0{||}H_1$) [$H_0$ \\ \color{blue600}H($a_0{||}a_1$) [$a_0$] [$a_1$]] [$H_1$ \\ \color{blue600}H($a_2{||}a_3$) [$a_2$] [$a_3$]]]
        [$H_5$ \\ \color{blue600}H($H_2{||}H_3$) [$H_2$ \\ \color{blue600}H($a_4{||}a_5$) [$a_4$] [$a_5$]] [$H_3$ \\ \color{blue600}H($H_6{||}H_7$) [$a_6$] [$a_7$]]]
        ]
    \end{forest}
\end{center}

\bigbreak

\section{LeanIMT}

\subsection{Definition}

\bigbreak

The \textbf{LeanIMT} (Lean Incremental Merkle Tree) is a Binary IMT.

\bigbreak

% align to the left
\raggedright

The LeanIMT has two properties:

1. Every node with two children is the hash of its two child nodes.

2. Every node with one child has the same value as its child node.

\bigbreak

Example of a LeanIMT

\bigbreak

$T = (V,E)$

% align to the left
\raggedright

\bigbreak

$V = \{a_0, a_1, a_2, H_0, H_1, H_2\}$

\bigbreak

$E = \{(a_0, H_0), (a_1, H_0), (a_2, a_2), (H_0, H_1), (a_2, a_2)\}$

\bigbreak

\begin{center}
    \begin{forest}
        for tree={edge path={\noexpand\path[\forestoption{edge}] (\forestOve{\forestove{@parent}}{name}.parent anchor) -- +(0,-18pt)-| (\forestove{name}.child anchor)\forestoption{edge label};}, rectangle, draw, align=center}
        [$H_1$ \\ \color{blue600}H($H_0{||}a_2$)
        [$H_0$ \\ \color{blue600}H($a_0{||}a_1$)
        [$a_0$]
            [$a_1$]
        ]
        [$a_2$ \\ \color{blue600}$a_2$
        [$a_2$]
        ]
        ]
    \end{forest}
\end{center}

\bigbreak

\subsection{Insertion}

$T = (V,E)$

% align to the left
\raggedright

\bigbreak

$V = \{a_0, a_1, a_2, a_3, H_0, H_1, H_2\}$

\bigbreak

$E = \{(a_0, H_0), (a_1, H_0), (a_2, H_1), (a_3, H_1), (H_0, H_2), (H_1, H_2)\}$

\bigbreak

\begin{multicols}{2}
    \vfill
    \columnbreak
    \vspace*{\fill}
    \begin{center}
        \begin{forest}
            for tree={edge path={\noexpand\path[\forestoption{edge}] (\forestOve{\forestove{@parent}}{name}.parent anchor) -- +(0,-12pt)-| (\forestove{name}.child anchor)\forestoption{edge label};}, rectangle, draw, align=center}
            [$H_2$
            [$H_0$
                    [$a_0$]
                        [$a_1$]
                ]
                [$H_1$
                    [$a_2$]
                        [$a_3$]
                ]
            ]
        \end{forest}
    \end{center}
    \begin{center}
        \textit{Before inserting $a_4$}
    \end{center}
    \begin{center}
        \begin{forest}
            for tree={edge path={\noexpand\path[\forestoption{edge}] (\forestOve{\forestove{@parent}}{name}.parent anchor) -- +(0,-12pt)-| (\forestove{name}.child anchor)\forestoption{edge label};}, rectangle, draw, align=center}
            [$H_3$, draw=blue600, top color=blue600!0, bottom color=blue600!0
            [
            $H_2$ [$H_0$ [$a_0$] [$a_1$]] [$H_1$ [$a_2$] [$a_3$]]
            ]
            [$a_4$ , edge=blue600, draw=blue600, top color=blue600!0, bottom color=blue600!0 [$a_4$ , edge=blue600, draw=blue600, top color=blue600!0, bottom color=blue600!0 [$a_4$ , edge=blue600, draw=blue600, top color=blue600!0, bottom color=blue600!0]]]
            ]
        \end{forest}
    \end{center}
    \begin{center}
        \textit{After inserting $a_4$}
    \end{center}
\end{multicols}

\bigbreak

\section{Benchmarks}

\section{Conslusions}

% Example citation in the Introduction section
This document is based on the work of \cite{semaphorev1whitepaper}.

% Allow LaTeX to break lines more flexibly in the bibliography
\sloppy

% Bibliography section
\printbibliography

% This is the end of the document
\end{document}